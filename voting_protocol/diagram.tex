\documentclass{standalone}
\usepackage[utf8]{inputenc}
\usepackage{polski}
\usepackage{tikz}
\usepackage{amsmath}
\usepackage{amssymb}
\usepackage{amsfonts}
\usepackage{enumitem}

\usepackage{ragged2e}
\newcommand\jtem{\item\justifying}

\begin{document}
	\begin{tikzpicture}

		\node at (-1,.5){};
		\foreach \x in {0,1,...,25}{
			\coordinate (ballot\x) at (0,-\x);
			\coordinate (client\x) at (10,-\x);
			\coordinate (voter\x) at (20,-\x);
			\coordinate (ballotEND) at (0,-\x);
			\coordinate (clientEND) at (10,-\x);
			\coordinate (voterEND) at (20,-\x);
			\node at (21,-\x){};
		}

		\draw[color=red!16] (ballot0) edge (ballotEND);
		\draw[color=green!16] (client0) edge (clientEND);
		\draw[color=blue!16] (voter0) edge (voterEND);

		\draw (ballot0) node[fill=red!16](ballot) {\textsc{ballot}};
		\draw (client0) node[fill=green!16](client) {\textsc{client}};
		\draw (voter0) node[fill=blue!16](voter) {\textsc{voter}};


		\draw (client1) edge node[sloped,above]
		{$\displaystyle
			\left(
				\#V,\#O
			\right)
				\in
				\mathbb{N}^2
		$} (ballot2);

		\draw (ballot3) edge node[sloped,above]
		{$\displaystyle
			\left(
				p,g
			\right)
				\in
				\mathbb{P} \times \mathbb{Z}_p
		$} (client4);

		\draw (client5) node[anchor=west,align=left]
		{
			$
				k \in \mathbb{Z}_p
			$
			\\
			$\displaystyle
				\left\{
					\left(t_v,y_v\right)|
					\gcd\left(y_v,p-1\right)=1
				\right\}
					_{v \in V}
					\subset
					\mathbb{Z}_p \times \mathbb{Z}_{p-1}
			$
		};

		\draw (client6) edge node[sloped,above] 
		{$\displaystyle
			\left\{
				g^{t_v}
			\right\}
				_{v \in V}
				\subset
				\mathbb{Z}_p
		$} (ballot7);

		\draw (client7) edge node[sloped,above]
		{$\displaystyle
			\left\{
				\left(
					g^{t_v},
					P_{o}g^{kt_v},
					g^{y_v},
					\left(P_{o}g^{kt_v}-t_vg^{y_v}\right)y_v^{-1}
				\right)
			\right\}
				_{o \in O}
				\subset
				\mathbb{Z}_p^3 \times \mathbb{Z}_{p-1}
		$} (voter8);

		\draw (voter9) node[anchor=east]
		{$\displaystyle
			c_v \in O
		$};

		\draw (voter10) edge node[sloped,above]
		{$\displaystyle
			\left(
				g^{t_v},
				P_{c_v}g^{kt_v},
				g^{y_v},
				\left(P_{c_v}g^{kt_v}-t_vg^{y_v}\right)y_v^{-1}
			\right)
				\in
				\mathbb{Z}_p^3 \times \mathbb{Z}_{p-1}
		$} (ballot11);

		\draw (ballot12) node[anchor=west]
		{$\displaystyle
			\left\{g^{t_v}\right\}
			\times
			\mathbb{Z}_p
			\cap
			\Gamma
			\stackrel{?}{=}
			\emptyset
		$};

		\draw (ballot13) node[anchor=west]
		{$\displaystyle
			g^{\left({P_{c_v}g^{kt_v}}\right)}
			\stackrel{?}{\equiv}_p
			{\left(g^{t_v}\right)}^
			{\left(g^{y_v}\right)}
			{\left(g^{y_v}\right)}^
			{\left(\left(P_{c_v}g^{kt_v}-t_vg^{y_v}\right)y_v^{-1}\right)}
		$};

		\draw (ballot14) node[anchor=west]
		{$\displaystyle
			\Gamma
			\gets
			\Gamma
			\cup
			\left(g^{t_v},P_{c_v}g^{kt_v}\right)
		$};

		\draw (client15) edge node[sloped,above]
		{
			\texttt{koniec głosowania}
		} (ballot16);

		\draw (ballot17) edge node[sloped,above]
		{$\displaystyle
			\left(
				\prod_{{v}\in{V'}}{g^{t_v}}, 
				\prod_{{v}\in{V'}}{P_{c_v}g^{kt_v}}
			\right)
				\in
				\mathbb{Z}_p^2
		$} (client18);

		\draw (client19) node[anchor=east]
		{$\displaystyle
			\prod_{{v}\in{V'}}{P_{c_v}}
			\equiv_p
			\left(
				\prod_{{v}\in{V'}}g^{t_v}
			\right)
			^{-k}
			\left(
				\prod_{{v}\in{V'}}{P_{c_v}g^{kt_v}}
			\right)
		$};

		\draw (client21) node[anchor=east]
		{$\displaystyle
			\operatorname{factor}
			\left(
				\prod_{{v}\in{V'}}{P_{c_v}}
			\right)
		$};


		\draw (10.5,-11) node[anchor=north west,text width=250,fill=cyan!8]
		{\fontsize{8pt}{3pt}\selectfont
			\begin{enumerate}[leftmargin=2em,rightmargin=1em]
				\jtem \textsc{client} wysyła do \textsc{ballot}'a dwie liczby. $\#V$ określa ilość osób jaka jest dopuszczona do głosowania, natomiast $\#O$ mówi ile jest możliwości wyboru.
				\jtem \textsc{ballot} odsyła do \textsc{client}'a liczbę pierwszą $p$ określającą grupę $\mathbb{Z}_p$ oraz jej generator $g$.
				\jtem \textsc{client} losuje klucz głosowania $k$ oraz parę kluczy tymczasowych $(t_v,y_v)$ dla każdego głosującego $v$.
				\jtem \textsc{client} przesyła do \textsc{ballot}'a wszystkie klucze prywatne $g^t$.
				\jtem \textsc{client} dystrybuuje wszystkim \textsc{voter}'om \emph{karty to głosowania} składające się z $\#O$ czwórek wyliczanych indywidualnie dla każdego z nich. Pierwszym elementem każdej czwórki danego \textsc{voter}'a $v$ jest odpowiadający mu klucz publiczny $g^{t_v}$. Kolejnym elementem jest zaszyfrowana przy pomocy $(k,t_v)$ liczba pierwsza $P_o$ odpowiadająca jednemu z możliwych wyborów $o \in O$. Trzeci element krotki to $g^{y_v}$. Czwarty natomiast jest wynikiem podpisania przy pomocy $(t_v,y_v)$ drugiego elementu tejże czwórki.
				\jtem \textsc{voter} wybiera $c_v$ będące jednym z elementów $O$.
				\jtem \textsc{voter} wysyła do \textsc{ballot}'a czwórkę odpowiadającą swojemu wyborowi $c_v$.
				\jtem \textsc{ballot} sprawdza czy nie oddano już głosu z tej karty do głosowania $g^{t_v}$.
				\jtem \textsc{ballot} sprawdza poprawność podpisu złożonego pod głosem.
				\jtem \textsc{ballot} księguje oddanie głosu.
				\jtem \textsc{client} wysyła do \textsc{ballot}'a informacje o końcu głosowania.
				\jtem \textsc{ballot} odsyła do \textsc{client}'a parę iloczynów kluczy publicznych $g^{t_v}$ oraz zaszyfrowanych komunikatów $P_{c_v}g^{kt_v}$. $v$ w tym przypadku przebiega po zbiorze $V'$ będącym podzbiorem $V$ zawierającym tylko tych \textsc{voter}'ów, którzy oddali poprawne głosy.
				\jtem \textsc{client} odszyfrowuje iloczyn $P_{c_v}$ przy pomocy klucza $k$
				\jtem \textsc{client} dokonuje faktoryzacji $P_{c_v}$ otrzymując potęgi kolejnych liczb pierwszych, które odpowiadają oddanym głosom.
			\end{enumerate}
		};
		
		\draw (10.5,-1) node[anchor=north west,align=right,text width=230,fill=yellow!16,inner sep=10]
		{
			{\fontsize{11}{1}\selectfont\hfill Piotr Majkrzak, Szymon Nosal\\}
			\vspace{8pt}
			{\fontsize{23}{1}\fontshape{sl}\selectfont\hfill głosowanie internetowe\\}
			{\fontsize{21}{1}\fontshape{sl}\selectfont\color{black!64}\hfill szkic protokołu\\}
			{\fontsize{12}{1}\fontshape{sl}\selectfont\color{black!64}\hfill Kraków, \today}
		};

	\end{tikzpicture}
\end{document}
